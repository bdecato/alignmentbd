\documentclass[11pt]{article}

\usepackage{fullpage,times,namedplus}

\title{alignmentbd User Manual}
\author{Benjamin E. Decato}
\newcommand{\prog}[1]{\texttt{#1}}

%%% load AMS-Latex Package
\usepackage{amsmath,amsfonts,amsthm,amssymb,amsopn,bm}

%%% paper layout, stylistic etc
\usepackage{fullpage,times}
\usepackage[paper=letterpaper,margin=1in,includeheadfoot,footskip=0.25in,headsep=0.25in]{geometry}
\usepackage{subfigure}
\usepackage{graphicx}
\usepackage{url}
\usepackage[ruled,vlined]{algorithm2e}
\usepackage[usenames,dvipsnames]{color}
\usepackage{amssymb}
\usepackage[pdfborder={0 0 1},colorlinks=true,citecolor=black,plainpages=false]{hyperref}


%%% FORMAT_GUIDELINE:  The following is a list of customized math symbols to enforce notation consistency. Please feel free to use them
%%%%%%%%%%%%%%%%%%%%%%%%%
\newcommand{\vct}[1]{\boldsymbol{#1}} % vector
\newcommand{\mat}[1]{\boldsymbol{#1}} % matrix
\newcommand{\cst}[1]{\mathsf{#1}} % constant

\newcommand{\field}[1]{\mathbb{#1}}
\newcommand{\R}{\field{R}} % real domain
% \newcommand{\C}{\field{C}} % complex domain
\newcommand{\F}{\field{F}} % functional domain

\newcommand{\T}{^{\textrm T}} % transpose

%% operator in linear algebra, functional analysis
\newcommand{\inner}[2]{#1\cdot #2}
\newcommand{\norm}[1]{\left\|#1\right\|}
\newcommand{\twonorm}[1]{\|#1\|_2^2}
% operator in functios, maps such as M: domain1 --> domain 2
\newcommand{\Map}[1]{\mathcal{#1}}

% operator in probability: expectation, covariance,
\newcommand{\ProbOpr}[1]{\mathbb{#1}}
% independence
\newcommand\independent{\protect\mathpalette{\protect\independenT}{\perp}}
\def\independenT#1#2{\mathrel{\rlap{$#1#2$}\mkern2mu{#1#2}}}
% conditional independence
\newcommand{\cind}[3]{{#1} \independent{#2}\,|\,#3}
% conditional expectation
\newcommand{\cndexp}[2]{\ProbOpr{E}\,[ #1\,|\,#2\,]}

% operator in optimization
\DeclareMathOperator{\argmax}{arg\,max}
\DeclareMathOperator{\argmin}{arg\,min}
\newcommand{\todo}[1]{{\color{red}#1}}

% environment
\newtheorem{thm}{Theorem}
\renewcommand{\labelenumi}{(\alph{enumi})}

% convenience commands
\newcommand{\eat}[1]{}


%%%%%%%%%%%%%%%%%%%%%%%%%%%%%%%%%%%%%%%%%%%%%%%%%%%%%%%%%%%%%%%%%%%%%%%%%%%%%%%%

\begin{document}

\maketitle

\tableofcontents
\newpage

\section{Overview}

This repository contains three alignment algorithms for DNA sequence
alignment. \prog{galignbd} performs global alignment between two DNA 
sequences, \prog{banded\_galignbd} is a more efficient version that requires
less space, and \prog{malignbd} implements multiple sequence alignment. Below
are descriptions of each program, along with usage. All programs have minimal
I/O error checking, so correct specification of arguments is critical at this
early stage in development.

\section{Theoretical overview}

Many alignment algorithms rely on the notion of a scoring function, in which
the alignments are rewarded for correct matches and penalized for placing
gaps (called "indels" in molecular biology) or allowing a mismatch. The general
structure of a scoring function for position x in sequence 1 and y in sequence
2 is:

\begin{displaymath}
w{x\choose y} = \left\{
  \begin{array}{lr}
    1 & : x = y\\
    -1 & : x \neq y\\
    -2 & : \text{x or y indel}
    \end{array}
    \right.
\end{displaymath}

In each of the three algorithms implemented in \prog{alignmentbd}, the goal is
to maximize the score for the alignment between the sequences and then trace
the solution back from the score matrix. All three are famous dynamic
programming algorithms, which I will go into now.

\begin{enumerate}
\item{\bf{galignbd}:}

Global alignment between two sequences.  Given two sequences $U=u_1,u_2,...u_n$
and $V=v_1,v_2,...v_m$, define $S(i,j)=S(u_1...u_i,v_1...v_j)$. The goal is to
maximize the score function $S(n,m)$ using the following recursive definition:

\begin{displaymath}
S(i,j) = max\left\{
  \begin{array}{lr}
    S(i-1,j-1) + w{u_i\choose v_j}\\
    S(i-1,j) + w{u_i\choose -} \\
    S(i,j-1) + w{-\choose v_j}
    \end{array}
    \right.
\end{displaymath}

Starting with the base case $S(0,0)=0$, \prog{galignbd} creates a table of
optimal alignment scores for subsets of the two strings. The entry in the
$(n,m)^{th}$ cell corresponds to the maximum global alignment score, and the
table is traced backwards to find the true alignment.

The time and space complexity for this algorithm as implemented are both
quadratic $\mathcal{O}(nm)$, as I iterate over all substrings computing their
score and storing it in an n by m table.  It is possible to perform global
alignment in linear space, but it is not implemented here: keep an eye
out for future updates!

\item{\bf{lalignbd}:}

Local alignment between two sequences. This program operates almost
identically to \prog{galignbd}, with a few crucial differences, the
most important occurring in the recurrence relation:

\begin{displaymath}
S(i,j) = max\left\{
  \begin{array}{lr}
    S(i-1,j-1) + w{u_i\choose v_j}\\
    S(i-1,j) + w{u_i\choose -} \\
    S(i,j-1) + w{-\choose v_j} \\
    0
    \end{array}
    \right.
\end{displaymath}

The additional option of "starting over" the alignment provides a
way to align two fragments of the original sequences. This is useful
when the sizes of the fragments differ substantially, preventing a
"gappy" global alignment, or when two sequences are known to share
small homologous regions of similarity such as transcription factor
binding sites, orthologous genes, or retrotransposons.

Local alignment runs a little slower because the optimal score could
occur in any cell of the matrix, not just the $(n,m)^{th}$ position,
so during the trace back step must search every cell.


\item{\bf{banded\_galignbd}:}

This algorithm identifies the global alignment for two strings that differ
by at most $k$ mismatches or indels, and is referred to as banded because
the score matrix truncates all cells too far away from the theoretical
perfect alignment, resulting in a diagonal band data structure. The
recursive definition of $S(i,j)$ is identical to global alignment, but in
addition to $S(0,0)=0$ as a base case, $S(i,j)=-\infty$ if $j<L[i]$ or
$j>R[i]$ where L and R are the bend at the $i^{th}$ row. That is, the cells
outside the $k$ mismatch boundary have a score of $-\infty$.

This algorithm tightens the problem to only closely related sequences, but
provides a useful edge: if you know your sequences are fairly close, you can
save space and time complexity from quadratic to $\mathcal{O}(kn)$ where $n$
is the larger string.

\item{\bf{malignbd}:}

This is an approximation algorithm for multiple sequence alignment. It works
as a profile aligner, where sequences are iteratively merged together into
profiles, which are snapshots containing the proportion of each nucleotide
at each site in the aligned sequence. For example, position 1 of a profile
containing three sites could be $A=0.3333$, $C=0.3333$, $T=0$, $G=0$,
$-=0.3333$ where one sequence contains an A, another a C, and another a
deletion at that site.. Given $P=p_1,p_2,..,p_n$ and $Q=q_1,q_2,..,q_n$ 
where $P$ and $Q$ are profiles, the score for a site between the two is:

\begin{displaymath}
g{p_i\choose q_i} = \sum\limits_{k,l\in\{ACTG-\}}w(k,l) p_i[k] q_j[l]
\end{displaymath}

The optimal global alignment between profiles can be calculated using the
same dynamic programming recursion as in global alignment with this modified
score function:

\begin{displaymath}
S(p_i,q_j) = max\left\{
  \begin{array}{lr}
    S(p_{i-1},q_{j-1}) + g{p_i\choose q_j}\\
    S(p_{i-1},q_j) + g{p_i\choose -} \\
    S(p_i,q_{j-1}) + g{-\choose q_j}
    \end{array}
    \right.
\end{displaymath}

Order of profile alignment is chosen heuristically based on their distance
from each other, calculated by counting the number of mismatches when no
alignment is performed on the original sequences in a pairwise manner. After
a profile is created from two sequences, the distance from the profile to
all other sequences is the average of the distance from each of the profile's
aligned sequences to each of the others.

As implemented, \prog{malignbd} runs in $\mathcal{O}(knm)$ time where $k$ is
the number of sequences and $n,m$ are the sizes of the two largest sequences.
This approximation is a measurable speedup over the naive case of multiple
alignment, which operates in  $\mathcal{O}(n^k)$ time, where $n$ is the 
longest sequence.

\end{enumerate}

\section{Running the programs}

Each of the programs has the option to provide a unique scoring function.
This must be done completely or not at all, and can be passed as follows:

\begin{verbatim}
> galignbd -m 2 -s 1 -i 3 first_sequence.fasta second_sequence.fasta
> lalignbd -m 2 -s 1 -i 3 first_sequence.fasta second_sequence.fasta
> banded_galignbd -m 2 -s 1 -i 3 first_sequence.fasta second_sequence.fasta
> malignbd -m 2 -s 1 -i 3 all_sequences.fasta
\end{verbatim}

This will provide a score function that penalizes mismatches by $1$, indels
by $3$, and ups the score for a match by $2$. For \prog{galignbd},
\prog{lalignbd}, and 
\prog{banded\_galignbd}, the sequences should come in FASTA format in two
seperate files. For \prog{malignbd}, they should come in one FASTA file
seperated by a newline. There is currently very little I/O error checking:
a problem I expect to fix in the future. Therefore be very careful to specify
the correct arguments on the command line.


\end{document}













